\documentclass[../main.tex]{subfiles}

\lfoot{Objects}

\begin{document}
So far we have discussed some of the many types of object that exist in python. Lets look at some of them(by no means an exhaustive list):\\
\\
\begin{minipage}[t]{0.5\linewidth}
    \begin{itemize}
        \item Strings
        \item Numbers (integers, float, complex)
        \item Tuples (imutable)
    \end{itemize}
\end{minipage}
\hfill
\begin{minipage}[t]{0.5\linewidth}
    \begin{itemize}
        \item Lists (mutable)
        \item Dictionaries (mutable)
        \item Functions
    \end{itemize}
\end{minipage}

\vspace{.5cm}
Side-note: Variables\\
Variables are not objects themselves. Generally when people are speaking by the book they say a given variable points to an object. Though often we talk about a variable type. So generally variables assume the type they are pointing towards.\\
\\
How actions occure on the objects has been seen in a couple of different ways. The first is a function(itself an object) it is called with the object as a parameter. These functions can be complex or simple but the built in functions tend to work on most types of objects.\\
\begin{verbatim}
>>>string = 'spam'
>>>len(string)
4
\end{verbatim}
Other actions occur on objects and tend to be more specific than the functions above. These methods are called in the following manor and belong to a specific type of object. Such as the split method we called on strings.\\
\begin{verbatim}
str = 'We are the knights who say nee'
str_lst = str.split()
print(str_lst)

['We', 'are', 'the', 'knights', 'who', 'say', 'nee']
\end{verbatim}
One of the best things about python that allows the
\end{document}
